\documentclass[]{article}
\usepackage{lmodern}
\usepackage{amssymb,amsmath}
\usepackage{ifxetex,ifluatex}
\ifnum 0\ifxetex 1\fi\ifluatex 1\fi=0 % if pdftex
  \usepackage[T1]{fontenc}
  \usepackage[utf8]{inputenc}
\setcounter{secnumdepth}{0}
 \usepackage[table,xcdraw]{xcolor}
 \usepackage[margin=1.5in]{geometry}
 \usepackage[T1]{fontenc}
\usepackage[tableposition=top]{caption}
\usepackage{tabularx}
\usepackage{xcolor}
\usepackage{hyperref}
\hypersetup{
    colorlinks=true,
    linkcolor=blue,
    filecolor=magenta,      
    urlcolor=cyan,
}



\title{Specyfikacja implementacyjna projektu indywidualnego \textbf{AiSD GR1}}
\author{Hubert Nakielski}
\date{Listopad 2020}

\begin{document}
\maketitle

\section{Informacje ogólne}

\section{Opis modułów}

\subsection{Pakiet \textit{file}}
Pakiet ten będzie odpowiedzialny za wszystkie czynności związane z plikami wejściowymi i wyjściowymi. 

\section{Opis klas }
\subsection{Klasa \textit{ConfigurationIO}}
\subsubsection{loadFromFile(String filePath)}
Metoda odpowiedzialna za czytanie pliku wejściowego, używa metody parseLine().
\subsubsection{loadToFile(File file)}
Metoda odpowiedzialna za wpisywanie gotowej konfiguracji do pliku wyjściowego.
\subsubsection{parseLine(String line)}
Metoda czytająca pojedynczą linię w pliku.

\section{Testowanie}

\section{Diagram klas}

\subsection{Użyte narzędzia}

\subsection{Konwencja}

\subsection{Warunki brzegowe}

\end{document}