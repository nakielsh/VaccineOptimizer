\documentclass[]{article}
\usepackage{lmodern}
\usepackage{amssymb,amsmath}
\usepackage{ifxetex,ifluatex}
\ifnum 0\ifxetex 1\fi\ifluatex 1\fi=0 % if pdftex
  \usepackage[T1]{fontenc}
  \usepackage[utf8]{inputenc}
\setcounter{secnumdepth}{0}
 \usepackage[table,xcdraw]{xcolor}
 \usepackage[margin=1.5in]{geometry}
 \usepackage[T1]{fontenc}
\usepackage[tableposition=top]{caption}
\usepackage{tabularx}
\usepackage{xcolor}
\usepackage{hyperref}
\hypersetup{
    colorlinks=true,
    linkcolor=blue,
    filecolor=magenta,      
    urlcolor=cyan,
}



\title{Specyfikacja funkcjonalna projektu indywidualnego \textbf{AiSD GR1}}
\author{Hubert Nakielski}
\date{Listopad 2020}


\begin{document}
\maketitle

\section{Cel projektu}
Celem projektu jest minimalizacja kosztów sprzedaży szczepionek przy jednoczesnym
 zapewnieniu dostaw do wszystkich aptek. Projekt powinien dostarczać najbardziej opłacalną 
 dla danych aptek konfigurację.
 
\section{Dane wejściowe}
Na wejście powinien być dostarczany plik wejściowy w formacie o rozszerzeniu \textit{.inp}, podający informacje o: \\
	\begin{itemize}
		\item
		Producentach szczepionek \\
		(id producenta, nazwa, dzienna produkcja)\\
		\item
		Aptekach\\ (id apteki, nazwa, dzienne zapotrzebowanie)\\
		\item
		Połączeniach producentów i aptek \\(id producenta, id apteki, dzienna maksymalna liczba dostarczanych szczepionek, koszt 
		szczepionki [zł])
	\end{itemize}

\begin{table}[h!]
	\caption{Producenci }
	\makebox[\linewidth]{
	\begin{tabular}{|c|c|c|}
		\hline
		\rowcolor[HTML]{C0C0C0} 
		\textbf{id producenta} int  & \textbf{nazwa} String      & \textbf{dzienna produkcja}  int \\ \hline
		0               & BioTech 2.0             & 900                      \\ \hline
		1               & Eko Polska 2020         & 1300                     \\ \hline
		2               & Post-Covid Sp. z o.o.   & 1100                     \\ \hline
	\end{tabular}
	}
\end{table}


\begin{table}[hbt!]
	\caption{Apteki }
	\makebox[\linewidth]{
	\begin{tabular}{|c|c|c|}
		\hline
		\rowcolor[HTML]{C0C0C0} 
		\textbf{id apteki} int & \textbf{nazwa}  String                & \textbf{dzienne zapotrzebowanie} int \\ \hline
		0         & CentMedEko Centrala    & 450                     \\ \hline
		1         & CentMedEko             & 690                     \\ \hline
		2         & CentMedEko Nowogrodzka & 1200                    \\ \hline
	\end{tabular}
	}
\end{table}

\begin{table}[h!]
	\caption{Połączenia producentów i aptek}
	\makebox[\linewidth]{
	\begin{tabular}{|c|c|c|c|}
		\hline
		\rowcolor[HTML]{C0C0C0} 
		\textbf{id producenta} int & \cellcolor[HTML]{C0C0C0}\textbf{id apteki} int & \textbf{\begin{tabular}[c]{@{}c@{}}dzienna maksymalna liczba\\  dostarczanych szczepionek\end{tabular}}  int & \textbf{koszt szczepionki {[}zł{]}} double\\ \hline
		0                      & 0                                          & 800                                                          & 70.5                                \\ \hline
		0                      & 1                                          & 600                                                          & 70                                  \\ \hline
		0                      & 2                                          & 750                                                          & 90.99                               \\ \hline
		1                      & 0                                          & 900                                                          & 100                                 \\ \hline
		1                      & 1                                          & 600                                                          & 80                                  \\ \hline
		1                      & 2                                          & 450                                                          & 70                                  \\ \hline
		2                      & 0                                          & 900                                                          & 80                                  \\ \hline
		2                      & 1                                          & 900                                                          & 90                                  \\ \hline
		2                      & 2                                          & 300                                                          & 100                                 \\ \hline
	\end{tabular}
	}
\end{table}
.\\\\\\\\\\\\
 Dla powyższych danych $\Uparrow$  plik wejściowy powinien wyglądać następująco:\\\\
\# Producenci szczepionek (id | nazwa | dzienna produkcja)\\
0 | BioTech 2.0 | 900\\
1 | Eko Polska 2020 | 1300\\
2 | Post-Covid Sp. z o.o. | 1100\\
\# Apteki (id | nazwa | dzienne zapotrzebowanie)\\
0 | CentMedEko Centrala | 450\\
1 | CentMedEko 24h | 690\\
2 | CentMedEko Nowogrodzka | 1200\\
\# Połączenia producentów i aptek (id producenta | id apteki | dzienna maksymalna liczba dostarczanych szczepionek | koszt szczepionki [zł] )\\
0 | 0 | 800 | 70.5\\
0 | 1 | 600 | 70\\
0 | 2 | 750 | 90.99\\
1 | 0 | 900 | 100\\
1 | 1 | 600 | 80\\
1 | 2 | 450 | 70\\
2 | 0 | 900 | 80\\
2 | 1 | 900 | 90\\
2 | 2 | 300 | 100\\




\clearpage

\section{Dane wyjściowe}
Program na wyjściu przekazuje dane najbardziej opłacalnej konfiguracji połączeń pomiędzy producentami,
a aptekami. Wygenerowane dane będą przekazane w poniższym formacie:\\\\
\textit{Nazwa producenta} -> 
\textit{Nazwa apteki} [Koszt = 
\textit{ilość kupionych szczepionek} * 
\textit{cena jednej szczepionki} = 
\textit{koszt kupionych szczepionek w PLN}]\\
...\\
...\\
Opłaty całkowite: 
\textit{opłata całkowita w PLN}\\\\
\emph{Przykład:}\\
BioTech 2.0
\hspace{7mm} -> CentMedEko Centrala [Koszt = 300 * 70.5 = 21150 zł]\\
Eko Polska 2020  -> CentMedEko Centrala [Koszt = 150 * 100 = 15000 zł]\\
/*\\
...\\
kolejne wiersze opisujące ustalone połączenia pomiędzy producentami, a aptekami\\
...\\
*/\\
Opłaty całkowite: 36150 zł\\


\section{Struktura katalogów}
\textcolor{orange}{2020Z\_AiSD\_proj\_ind\_GR1\_gr19}

	|---\textcolor{orange}{src}
	
	|\hspace{4mm} |-----\textcolor{orange}{vaccine}

	|\hspace{15mm}|-----\textcolor{orange}{file}
	
	|\hspace{15mm}|-----\textcolor{orange}{objects}
	
	|\hspace{15mm}|-----\textcolor{orange}{calculations}
	
	|
	
	|
	
	|---\textcolor{orange}{test}
	
	\hspace{5mm} |-----\textcolor{orange}{vaccine} 

	\hspace{16mm}|-----\textcolor{orange}{file}
	
	\hspace{16mm}|-----\textcolor{orange}{calculations}

\clearpage

\section{Komunikaty błędów}
Program weryfikuje poprawność pliku wejściowego oraz danych wejściowych i wyświetla odpowiedni komunikat do wystąpionego błędu. \\ Błędy wykrywalne przez program:\\\\
- łączna dzienna produkcja szczepionek będzie mniejsza niż łączne dzienne zapotrzebowanie;\\
- wystąpi błąd w składni pliku wejściowego;\\
- liczba połączeń producentów i aptek nie jest równa iloczynowi ilości producentów i aptek;\\
- nazwy bądź id aptek się powtarzają;\\
- nazwy bądź id producentów się powtarzają.\\

\section{Uruchomienie programu}
Do uruchomienia programu potrzebne jest zainstalowane oprogramowanie Java 14.0.1, możliwe do pobrania \href{https://jdk.java.net/archive/}{tutaj}.\\\\
Aby uruchomić program należy, będąc w odpowiednim pliku obejmującym, wpisać dwie komendy w terminalu:\\
\begin{center}
\textit{javac Main.java\\
java Main.java <ścieżka do pliku wejściowego>}

\end{center}

\section{Testowanie}
Do testowania kodu użyję narzędzia JUnit, a działanie algorytmu liczącego przetestuję ręcznie.

\end{document}
